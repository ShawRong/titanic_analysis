\documentclass[conference]{IEEEtran}
\IEEEoverridecommandlockouts
% The preceding line is only needed to identify funding in the first footnote. If that is unneeded, please comment it out.
%Template version as of 6/27/2024

\usepackage{cite}
\usepackage{amsmath,amssymb,amsfonts}
\usepackage{algorithmic}
\usepackage{graphicx}
\usepackage{textcomp}
\usepackage{xcolor}
\def\BibTeX{{\rm B\kern-.05em{\sc i\kern-.025em b}\kern-.08em
    T\kern-.1667em\lower.7ex\hbox{E}\kern-.125emX}}
\begin{document}

\title{Conference Paper Title*\\
{\footnotesize \textsuperscript{*}Note: Sub-titles are not captured for https://ieeexplore.ieee.org  and
should not be used}
\thanks{Identify applicable funding agency here. If none, delete this.}
}

\author{
\IEEEauthorblockN{1\textsuperscript{st} Shuo Rong}
\IEEEauthorblockA{
\textit{shawnrong1213@gmail.com}
\textit{21126613}
}
\and
\IEEEauthorblockN{1\textsuperscript{st} Shuo Rong}
\IEEEauthorblockA{
\textit{shawnrong1213@gmail.com}
\textit{21126613}
}
\and
\IEEEauthorblockN{1\textsuperscript{st} Shuo Rong}
\IEEEauthorblockA{
\textit{shawnrong1213@gmail.com}
\textit{21126613}
}
}

\maketitle

\begin{abstract}
This document is a model and instructions for \LaTeX.
This and the IEEEtran.cls file define the components of your paper [title, text, heads, etc.]. *CRITICAL: Do Not Use Symbols, Special Characters, Footnotes, 
or Math in Paper Title or Abstract.
\end{abstract}

\begin{IEEEkeywords}
component, formatting, style, styling, insert.
\end{IEEEkeywords}

\section{Introduction}

\subsection{Background}

The sinking of the Titanic on April 15, 1912, is one of history's most notorious maritime disasters. During its maiden voyage from Southampton to New York City, the RMS Titanic struck an iceberg, leading to the tragic loss of around 1,502 of the 2,224 passengers and crew aboard \cite{kaggleTitanicMachine}. This event not only shocked the world but also revealed deep social and economic disparities of the time.

The Titanic was a microcosm of early 20th-century society, carrying both some of the wealthiest individuals and hundreds of emigrants seeking new opportunities in North America.\cite{wiki:Titanic} The contrast between social classes aboard the ship was stark, with first-class passengers enjoying luxurious accommodations and proximity to lifeboats, while many third-class passengers faced dire conditions below deck.

Survival rates varied dramatically based on social class. Approximately 62\% of first-class passengers survived, compared to only 25\% of third-class passengers\cite{wiki:Titanic}. Many third-class individuals were trapped in the ship’s labyrinthine corridors, struggling to reach the lifeboats in time. Second-class passengers had a survival rate of about 42\%\cite{wiki:Titanic}, but they were still at a disadvantage compared to the wealthier elite.

Gender also influenced survival outcomes, particularly through the implementation of the "women and children first" protocol. While this maritime tradition aimed to prioritize the safety of women and children, it was inconsistently applied during the disaster. First-class women had a survival rate of 9\%, while only 47\% of third-class women survived\cite{wiki:Titanic}. Children in lower classes faced grim odds as well, with many unable to escape alongside their families due to delayed evacuation efforts and inadequate communication about the severity of the situation.

The tragedy of the Titanic serves as a poignant reminder of the inequalities that can persist even in life-or-death scenarios. The disaster not only highlighted the devastating impact of the sinking but also reflected the rigid social hierarchies of the Edwardian era, where wealth and privilege often dictated one’s fate.

\subsection{Problem Definition}

The Kaggle competition "Titanic - Machine Learning from Disaster" challenges participants to build a predictive model to determine "What types of people were more likely to survive the Titanic disaster?"\cite{kaggleTitanicMachine} 

To enhance model performance, a systematic approach to data preprocessing and feature engineering is essential. This involves handling missing values by imputing passenger ages using strategies like mean or median imputation, regression predictions, or clustering, while also addressing other gaps such as missing embarkation points or fares. Feature engineering plays a crucial role, allowing for the creation of new variables based on domain knowledge, such as family size, ticket group size, and cabin presence. Categorical variables, like gender and embarkation points, should be transformed into numerical formats through one-hot encoding or label encoding. Continuous features, including age and fare, can be grouped and binned to reveal hidden patterns. 

Visualizing data helps explore correlations and distributions, providing insights into relationships between survival and factors like age, gender, and class. It's also important to identify and manage outliers or anomalies within the dataset. 

A comparison of various algorithms, such as Logistic Regression, Random Forests, Gradient Boosted Trees, and Neural Networks, will help identify the most effective approach. Implementing cross-validation ensures the model's robustness, while hyperparameter tuning optimizes performance. Finally, assessing model performance through metrics like accuracy, precision, recall, and the F1-score will provide a comprehensive understanding of the model's effectiveness.

This holistic strategy combines careful data handling, innovative feature creation, and advanced modeling techniques to achieve higher accuracy and deliver meaningful insights into the factors influencing survival.

\subsection{Related Work}

The blog post Data Quality and Being Curious: Titanic by DataBooth (2021) provides an insightful critique of the Titanic dataset, widely utilized in data science education and Kaggle competitions\cite{databoothDataQuality}. The author identifies and analyzes discrepancies within the dataset, such as an erroneously included 80-year-old passenger whose recorded birthdate postdates the Titanic disaster. This anomaly underscores potential systematic errors in the dataset, particularly in age reporting\cite{databoothDataQuality}. The analysis highlights the importance of data verification and quality assurance in machine learning applications, cautioning that such discrepancies are often overlooked by users. This critique supports the rationale for integrating or substituting alternative datasets to enhance reliability and accuracy in analytical models.

The Encyclopedia Titanica\cite{encyclopediatitanicaEncyclopediaTitanica} serves as a comprehensive resource for information related to the RMS Titanic, offering a wealth of historical data, personal accounts, and scholarly analyses. This online platform compiles extensive research on the ship's construction, its tragic sinking, and its cultural impact, making it an invaluable reference for scholars and enthusiasts alike. The site includes detailed biographies of passengers and crew, as well as a plethora of articles discussing various aspects of the Titanic's legacy. Its commitment to preserving and disseminating knowledge about this significant maritime disaster underscores its importance in the field of historical research.

The Wikipedia page Passengers\cite{wiki:Passengers_of_the_Titanic} of the Titanic provides a detailed dataset of the individuals who traveled on the RMS Titanic, categorizing them into three distinct classes. First Class was primarily composed of affluent and influential individuals, while Second Class represented middle-class travelers. Third Class largely consisted of immigrants seeking new opportunities in North America\cite{wiki:Passengers_of_the_Titanic}.
The dataset includes various attributes, such as the passengers' names, ages, hometowns, ports of embarkation, intended destinations, lifeboat assignments, and body recovery information. This rich collection offers valuable insights into the demographics and experiences of the passengers, making it an excellent resource for extending datasets. Utilizing this information can help incorporate additional features and enhance the predictive performance of related models.

The Kaggle Titanic Competition\cite{kaggleTitanicMachine} is a well-known introductory challenge in machine learning, aimed at helping participants learn predictive modeling techniques. The objective is to predict the survival of passengers aboard the RMS Titanic using historical data. Participants receive a labeled dataset that includes features such as passenger demographics (age, gender, class), family size, and fare paid.
This challenge is framed as a binary classification task, where the target variable indicates whether a passenger survived or did not survive. It emphasizes essential preprocessing techniques, such as handling missing data and feature engineering, as well as model evaluation.
Overall, the competition is highly regarded as a foundational exercise for gaining practical experience in data exploration, feature extraction, and model evaluation\cite{kaggleTitanicMachine}.

\section{Design and Implementation}

\subsection{Data Preprocessing}
The first and foremost task is to identify materials that can provide additional insights into the passengers and help the model to predict their survival well. Initially, we utilized data from Wikipedia to enhance the original dataset obtained from Kaggle competition. During our exploration, we discovered some extended dataset that integrates Wikipedia information. One of them was particularly beneficial because the original Kaggle dataset did not contain complete survival labels for all passengers; some of these labels were not provived because it's testing dataset, which meant they are not available.

By merging these datasets, we were able to create an enriched dataset that filled in crucial gaps. This new dataset includes complete age information (with no missing values, unlike the original), as well as details about passengers' hometowns (broken down by country and region), their destinations, lifeboat assignments, and body identification numbers indicating the recovery vessel.

From the original dataset, we derived additional features, such as a family size metric, which combines the number of siblings/spouses (SibSp) and parents/children (Parch). However, we encountered challenges with the 'hometown' feature, as it was not a well-defined label. The complexity arose from the fact that it encompassed various geographical levels, from countries down to specific towns, making it a less-than-ideal categorical feature for analysis. To address this, we manually designed code to parse and categorize the hometown information into distinct country and region labels.

Furthermore, we utilized the dataset with full survival labels to supplement the original dataset. This integration process required careful alignment of records based on passenger names, which involved handling complex name variations and the potential for name duplications. Additionally, we noted discrepancies in features related to the same individual; for instance, a passenger might have different embarkation locations listed across datasets. To resolve these inconsistencies, we conducted thorough investigations into the historical context surrounding these discrepancies.

Through this meticulous process, we aimed to build a robust and comprehensive dataset that not only enhances our understanding of the passengers but also facilitates more accurate analyses and insights into their survival outcomes.

\subsection{ Overview of Solution}
decision tree
random forest
etc.

\subsection{The Insight of Solution} 

(e.g. preprocessing, feature engineering, model (including the data mining/machine learning models) etc.)

\subsection{Implementation Details}



\section{Prepare Your Paper Before Styling}
Before you begin to format your paper, first write and save the content as a 
separate text file. Complete all content and organizational editing before 
formatting. Please note sections \ref{AA} to \ref{FAT} below for more information on 
proofreading, spelling and grammar.

Keep your text and graphic files separate until after the text has been 
formatted and styled. Do not number text heads---{\LaTeX} will do that 
for you.

\subsection{Abbreviations and Acronyms}\label{AA}
Define abbreviations and acronyms the first time they are used in the text, 
even after they have been defined in the abstract. Abbreviations such as 
IEEE, SI, MKS, CGS, ac, dc, and rms do not have to be defined. Do not use 
abbreviations in the title or heads unless they are unavoidable.

\subsection{Units}
\begin{itemize}
\item Use either SI (MKS) or CGS as primary units. (SI units are encouraged.) English units may be used as secondary units (in parentheses). An exception would be the use of English units as identifiers in trade, such as ``3.5-inch disk drive''.
\item Avoid combining SI and CGS units, such as current in amperes and magnetic field in oersteds. This often leads to confusion because equations do not balance dimensionally. If you must use mixed units, clearly state the units for each quantity that you use in an equation.
\item Do not mix complete spellings and abbreviations of units: ``Wb/m\textsuperscript{2}'' or ``webers per square meter'', not ``webers/m\textsuperscript{2}''. Spell out units when they appear in text: ``. . . a few henries'', not ``. . . a few H''.
\item Use a zero before decimal points: ``0.25'', not ``.25''. Use ``cm\textsuperscript{3}'', not ``cc''.)
\end{itemize}

\subsection{Equations}
Number equations consecutively. To make your 
equations more compact, you may use the solidus (~/~), the exp function, or 
appropriate exponents. Italicize Roman symbols for quantities and variables, 
but not Greek symbols. Use a long dash rather than a hyphen for a minus 
sign. Punctuate equations with commas or periods when they are part of a 
sentence, as in:
\begin{equation}
a+b=\gamma\label{eq}
\end{equation}

Be sure that the 
symbols in your equation have been defined before or immediately following 
the equation. Use ``\eqref{eq}'', not ``Eq.~\eqref{eq}'' or ``equation \eqref{eq}'', except at 
the beginning of a sentence: ``Equation \eqref{eq} is . . .''

\subsection{\LaTeX-Specific Advice}

Please use ``soft'' (e.g., \verb|\eqref{Eq}|) cross references instead
of ``hard'' references (e.g., \verb|(1)|). That will make it possible
to combine sections, add equations, or change the order of figures or
citations without having to go through the file line by line.

Please don't use the \verb|{eqnarray}| equation environment. Use
\verb|{align}| or \verb|{IEEEeqnarray}| instead. The \verb|{eqnarray}|
environment leaves unsightly spaces around relation symbols.

Please note that the \verb|{subequations}| environment in {\LaTeX}
will increment the main equation counter even when there are no
equation numbers displayed. If you forget that, you might write an
article in which the equation numbers skip from (17) to (20), causing
the copy editors to wonder if you've discovered a new method of
counting.

{\BibTeX} does not work by magic. It doesn't get the bibliographic
data from thin air but from .bib files. If you use {\BibTeX} to produce a
bibliography you must send the .bib files. 

{\LaTeX} can't read your mind. If you assign the same label to a
subsubsection and a table, you might find that Table I has been cross
referenced as Table IV-B3. 

{\LaTeX} does not have precognitive abilities. If you put a
\verb|\label| command before the command that updates the counter it's
supposed to be using, the label will pick up the last counter to be
cross referenced instead. In particular, a \verb|\label| command
should not go before the caption of a figure or a table.

Do not use \verb|\nonumber| inside the \verb|{array}| environment. It
will not stop equation numbers inside \verb|{array}| (there won't be
any anyway) and it might stop a wanted equation number in the
surrounding equation.

\subsection{Some Common Mistakes}\label{SCM}
\begin{itemize}
\item The word ``data'' is plural, not singular.
\item The subscript for the permeability of vacuum $\mu_{0}$, and other common scientific constants, is zero with subscript formatting, not a lowercase letter ``o''.
\item In American English, commas, semicolons, periods, question and exclamation marks are located within quotation marks only when a complete thought or name is cited, such as a title or full quotation. When quotation marks are used, instead of a bold or italic typeface, to highlight a word or phrase, punctuation should appear outside of the quotation marks. A parenthetical phrase or statement at the end of a sentence is punctuated outside of the closing parenthesis (like this). (A parenthetical sentence is punctuated within the parentheses.)
\item A graph within a graph is an ``inset'', not an ``insert''. The word alternatively is preferred to the word ``alternately'' (unless you really mean something that alternates).
\item Do not use the word ``essentially'' to mean ``approximately'' or ``effectively''.
\item In your paper title, if the words ``that uses'' can accurately replace the word ``using'', capitalize the ``u''; if not, keep using lower-cased.
\item Be aware of the different meanings of the homophones ``affect'' and ``effect'', ``complement'' and ``compliment'', ``discreet'' and ``discrete'', ``principal'' and ``principle''.
\item Do not confuse ``imply'' and ``infer''.
\item The prefix ``non'' is not a word; it should be joined to the word it modifies, usually without a hyphen.
\item There is no period after the ``et'' in the Latin abbreviation ``et al.''.
\item The abbreviation ``i.e.'' means ``that is'', and the abbreviation ``e.g.'' means ``for example''.
\end{itemize}
An excellent style manual for science writers is \cite{b7}.

\subsection{Authors and Affiliations}\label{AAA}
\textbf{The class file is designed for, but not limited to, six authors.} A 
minimum of one author is required for all conference articles. Author names 
should be listed starting from left to right and then moving down to the 
next line. This is the author sequence that will be used in future citations 
and by indexing services. Names should not be listed in columns nor group by 
affiliation. Please keep your affiliations as succinct as possible (for 
example, do not differentiate among departments of the same organization).

\subsection{Identify the Headings}\label{ITH}
Headings, or heads, are organizational devices that guide the reader through 
your paper. There are two types: component heads and text heads.

Component heads identify the different components of your paper and are not 
topically subordinate to each other. Examples include Acknowledgments and 
References and, for these, the correct style to use is ``Heading 5''. Use 
``figure caption'' for your Figure captions, and ``table head'' for your 
table title. Run-in heads, such as ``Abstract'', will require you to apply a 
style (in this case, italic) in addition to the style provided by the drop 
down menu to differentiate the head from the text.

Text heads organize the topics on a relational, hierarchical basis. For 
example, the paper title is the primary text head because all subsequent 
material relates and elaborates on this one topic. If there are two or more 
sub-topics, the next level head (uppercase Roman numerals) should be used 
and, conversely, if there are not at least two sub-topics, then no subheads 
should be introduced.

\subsection{Figures and Tables}\label{FAT}
\paragraph{Positioning Figures and Tables} Place figures and tables at the top and 
bottom of columns. Avoid placing them in the middle of columns. Large 
figures and tables may span across both columns. Figure captions should be 
below the figures; table heads should appear above the tables. Insert 
figures and tables after they are cited in the text. Use the abbreviation 
``Fig.~\ref{fig}'', even at the beginning of a sentence.

\begin{table}[htbp]
\caption{Table Type Styles}
\begin{center}
\begin{tabular}{|c|c|c|c|}
\hline
\textbf{Table}&\multicolumn{3}{|c|}{\textbf{Table Column Head}} \\
\cline{2-4} 
\textbf{Head} & \textbf{\textit{Table column subhead}}& \textbf{\textit{Subhead}}& \textbf{\textit{Subhead}} \\
\hline
copy& More table copy$^{\mathrm{a}}$& &  \\
\hline
\multicolumn{4}{l}{$^{\mathrm{a}}$Sample of a Table footnote.}
\end{tabular}
\label{tab1}
\end{center}
\end{table}

\begin{figure}[htbp]
\centerline{\includegraphics{fig1.png}}
\caption{Example of a figure caption.}
\label{fig}
\end{figure}

Figure Labels: Use 8 point Times New Roman for Figure labels. Use words 
rather than symbols or abbreviations when writing Figure axis labels to 
avoid confusing the reader. As an example, write the quantity 
``Magnetization'', or ``Magnetization, M'', not just ``M''. If including 
units in the label, present them within parentheses. Do not label axes only 
with units. In the example, write ``Magnetization (A/m)'' or ``Magnetization 
\{A[m(1)]\}'', not just ``A/m''. Do not label axes with a ratio of 
quantities and units. For example, write ``Temperature (K)'', not 
``Temperature/K''.

\section*{Acknowledgment}

The preferred spelling of the word ``acknowledgment'' in America is without 
an ``e'' after the ``g''. Avoid the stilted expression ``one of us (R. B. 
G.) thanks $\ldots$''. Instead, try ``R. B. G. thanks$\ldots$''. Put sponsor 
acknowledgments in the unnumbered footnote on the first page.

\section*{References}

Please number citations consecutively within brackets \cite{b1}. The 
sentence punctuation follows the bracket \cite{b2}. Refer simply to the reference 
number, as in \cite{b3}---do not use ``Ref. \cite{b3}'' or ``reference \cite{b3}'' except at 
the beginning of a sentence: ``Reference \cite{b3} was the first $\ldots$''

Number footnotes separately in superscripts. Place the actual footnote at 
the bottom of the column in which it was cited. Do not put footnotes in the 
abstract or reference list. Use letters for table footnotes.

Unless there are six authors or more give all authors' names; do not use 
``et al.''. Papers that have not been published, even if they have been 
submitted for publication, should be cited as ``unpublished'' \cite{b4}. Papers 
that have been accepted for publication should be cited as ``in press'' \cite{b5}. 
Capitalize only the first word in a paper title, except for proper nouns and 
element symbols.

For papers published in translation journals, please give the English 
citation first, followed by the original foreign-language citation \cite{b6}.

\bibliographystyle{plain}
\bibliography{tex}

\vspace{12pt}
\color{red}
IEEE conference templates contain guidance text for composing and formatting conference papers. Please ensure that all template text is removed from your conference paper prior to submission to the conference. Failure to remove the template text from your paper may result in your paper not being published.

\end{document}
